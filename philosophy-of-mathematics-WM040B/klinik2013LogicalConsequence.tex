\documentclass[a4paper]{article}

% Font stuff
\usepackage{fouriernc}
%\usepackage{mathptmx}
\usepackage[T1]{fontenc}

\usepackage[utf8]{inputenc}

% Extra math symbols
\usepackage{amsmath}
\usepackage{amssymb}
\usepackage{amsthm}

% IEEE equation array
\usepackage[retainorgcmds]{IEEEtrantools}

% bibliography
%\usepackage[square]{natbib}

% adds a toc to the pdf file, and makes refs clickable
\usepackage[bookmarks,colorlinks=true]{hyperref}

% for \llbracket, \rrbracket (scott brackets)
% Needs debian package texlive-math-extra
\usepackage{stmaryrd}

% === BEGIN custom commands

\newcommand{\arr}{\rightarrow}
\newcommand{\todo}[1]{\bigskip \noindent \emph{todo: #1}}
%\newcommand{\todo}[1]{}
\newcommand{\semantics}[1]{\llbracket #1 \rrbracket}

% church-style (explicit) abstraction. We adjust the spacing around the colon and dot
\newcommand{\church}[4]{#1 #2\!:\!#3\,.\,#4}

\newcommand{\curry}[3]{#1 #2\,.\,#3}

% something is of type something  x : A, we adjust the spacing
\newcommand{\oftype}[2]{#1\!:\!#2}

% === END custom commands

\begin{document}

\title{On Logical Consequence}
\author{Markus Klinik (s4220315)}
\maketitle

This document is a brief summary of two papers on logical consequence: Tarski's
``On The Concept of Following Logically'' and Shapiro's ``Logical Consequence,
Proof Theory and Model Theory''.

The summary of the original texts is given in normal font.  My questions and
comments are set in italics.

\section{Tarksi}

Making precise the notion of \emph{following logically} was made in the past in
an arbitrary manner.

\subsection{The syntactical approach}

Deductive system:

\begin{itemize}
    \item Start from axioms
    \item Apply inference rules
    \item Accept conclusion as proven
\end{itemize}

\emph{Which formal system is he referring to?  Hilbert?}

Logicians suppose that these few rules of inference completely capture the
notion of following.  Their point is justified by sucessfully proving lots of
old theorems.

\subsection{Criticism of the syntactic approach}

Counterexample: In the deductive system, it is not the
case that from the infinite set of axioms $\{ n\text{ has property }P\ |\ n \in
\mathbb{N} \}$ the sentence ``Every number has property $P$'' follows.  But this
is indubitable the case from the viewpoint of everyday intuition.  Systems
exhibiting this shortcoming are called $\omega$-incomplete.

We can try to fix the situation by adding the principle described above as a
rule in the system.  Only now, this rule has infinitely many premises, which
would make our proof trees infinitary.  How can one prove infinitely many
premises?

We need some kind of meta-argument: all these sentences are provable.  Note the
difference between ``is provable'' and ``has been proven''.  Now, ``is provable
in the system'' is a sentence in metatheory.  Thus, our deductive system must be
rich enough to talk about itself.  This is easy.  Any system where arithmetic is
possible can talk about itself.  At first sight, enriching our system with
meta-arguments seems to sufficiently capture what was missing.

Enter Gödel: If the system is rich enough, we can always construct sentences
that follow in the everyday sense, but cannot be proven in the system.

Therefore, we need a different conceptual apparatus than deductive systems.
Nota bene: the $\omega$-incomplete system is still important, because it works
well for what it does.

\section{Shapiro}

\bibliographystyle{plain}
\bibliography{logic}

\end{document}

% vim: textwidth=80
