\documentclass[a4paper]{article}

% Font stuff
\usepackage{fouriernc}
%\usepackage{mathptmx}
\usepackage[T1]{fontenc}

\usepackage[utf8]{inputenc}

% Extra math symbols
\usepackage{amsmath}
\usepackage{amssymb}
\usepackage{amsthm}

% IEEE equation array
\usepackage[retainorgcmds]{IEEEtrantools}

% bibliography
%\usepackage[square]{natbib}

% adds a toc to the pdf file, and makes refs clickable
\usepackage[bookmarks,colorlinks=true]{hyperref}

% for \llbracket, \rrbracket (scott brackets)
% Needs debian package texlive-math-extra
\usepackage{stmaryrd}

% === BEGIN custom commands

\newcommand{\arr}{\rightarrow}
\newcommand{\Arr}{\Rightarrow}
\newcommand{\todo}[1]{\bigskip \noindent \emph{todo: #1}}
%\newcommand{\todo}[1]{}
\newcommand{\semantics}[1]{\llbracket #1 \rrbracket}

% church-style (explicit) abstraction. We adjust the spacing around the colon and dot
\newcommand{\church}[4]{#1 #2\!:\!#3\,.\,#4}

\newcommand{\curry}[3]{#1 #2\,.\,#3}

% something is of type something  x : A, we adjust the spacing
\newcommand{\oftype}[2]{#1\!:\!#2}

% === END custom commands

\begin{document}

\title{On Logical Consequence}
\author{Markus Klinik (s4220315)}
\maketitle

This document is a brief summary of two papers on logical consequence: Tarski's
``On The Concept of Following Logically'' and Shapiro's ``Logical Consequence,
Proof Theory and Model Theory''.

The summary of the original texts is given in normal font.  My questions and
comments are set in italics.

\section{Tarksi}

Making precise the notion of \emph{following logically} was made in the past in
an arbitrary manner.

\subsection{The syntactical approach}

Deductive system:

\begin{itemize}
    \item Start from axioms
    \item Apply inference rules
    \item Accept conclusion as proven
\end{itemize}

\emph{Which formal system is he referring to?  Hilbert?}

Logicians suppose that these few rules of inference completely capture the
notion of following.  Their point is justified by sucessfully proving lots of
old theorems.

\subsection{Criticism of the syntactic approach}

Counterexample: In the deductive system, it is not the
case that from the infinite set of axioms $\{ n\text{ has property }P\ |\ n \in
\mathbb{N} \}$ the sentence ``Every number has property $P$'' follows.  But this
is indubitable the case from the viewpoint of everyday intuition.  Systems
exhibiting this shortcoming are called $\omega$-incomplete.

We can try to fix the situation by adding the principle described above as a
rule in the system.  Only now, this rule has infinitely many premises, which
would make our proof trees infinitary.  How can one prove infinitely many
premises?

We need some kind of meta-argument: all these sentences are provable.  Note the
difference between ``is provable'' and ``has been proven''.  Now, ``is provable
in the system'' is a sentence in metatheory.  Thus, our deductive system must be
rich enough to talk about itself.  This is easy.  Any system where arithmetic is
possible can talk about itself.  At first sight, enriching our system with
meta-arguments seems to sufficiently capture what was missing.

Enter Gödel: If the system is rich enough, we can always construct sentences
that follow in the everyday sense, but cannot be proven in the system.

Therefore, we need a different conceptual apparatus than deductive systems.
Nota bene: the $\omega$-incomplete system is still important, because it works
well for what it does.

\subsection{The semantic approach}

\emph{In the following, $\Gamma$ is always a set of sentences, and $\varphi$ a
sentence.}

First try, by Carnap: $\varphi$ is a consequence of $\Gamma$ iff $\Gamma$
together with $\neg\varphi$ leads to a contradiction.  This approach however is
too narrow, because it requires the concepts of contradiction and negation in
the language.

Let us first think about which intuitive properties the notion of consequence
should have.  Mainly two things:
\begin{enumerate}
  \item Assume $\varphi$ follows from $\Gamma$ intuitively.  It cannot be the
  case that every sentence in $\Gamma$ is true but $\varphi$ is false.

  \item Following cannot depend on our knowledge of the external world.  In
  particular, it cannot depend on the objects we are speaking of.
\end{enumerate}
These conditions are necessary but not sufficient:  The language would be
required to have a constant symbol for every object.  \emph{I don't understand
why that is.}

Introduce the notion of model:  Let $\Gamma$ be a theory. Replace all constant
symbols with variables, call this $\Gamma'$.  A \emph{model} of $\Gamma$ is a
sequence of objects satisfying every sentential function in $\Gamma'$.

Now we can formulate the famous definition: A sentence $\varphi$ follows
logically from a theory $\Gamma$ iff every model of $\Gamma$ is also a model of
$\varphi$.

Some Notes: This definition fulfils the condition from above.  The definition is
independent of the language under consideration.  If the language has negation,
and defining contradictory as having no model, this definition is equivalent to
the one by Carnap.

\subsection{An open question}

The definition relies on the distinction between logical and non-logical
symbols.  But where to draw the line?  At least implication and quantifier must
be logical for this to work, but it would make sense to regard all symbols as
logical.  In this case, Tarski's consequence would be equivalent to material
consequence.  \emph{How so? What does he mean by \emph{all symbols}? Certainly
not relations and functions?}

\section{Shapiro}

There are two ways logic is commonly researched.  Mathematical logic, in
particular proof theory, model theory, set theory and computational complexity.
The other way is philosophy of logic, in particular questions about correct
reasoning, valid thought and inference.

There are two precise notions of logical consequence: the deductive and the
semantic notion.

$\Gamma \vdash \varphi$ iff there is a sequence of formulas ending with
$\varphi$ such that every element of the sequence is either an element of
$\Gamma$ or is obtained by one of the inference rules from previous elements of
the sequence.

$\Gamma \models \varphi$ iff every model of $\Gamma$ is also a model of
$\varphi$.

The deductive notion is always relative to a deductive system.  The semantic
notion as always relative to a set theory.  The latter relation is always
implicit and therefore often forgotten.

These two notions are connected by soundness and completeness.

$\vdash\ \Arr\ \models$ is called ``soundness'' and is usually easy to prove.

$\models\ \Arr\ \vdash$ is called ``completeness'' and is usually hard to prove,
requires deep mathematical insight into the systems involved.

Our main concern here: what does the semantic notion of logical consequence have
to do correct reasoning?

\subsection{Logical consequence}

We begin with a survey of some of the proposals for pretheoretic, i.e. intuitive
notion of logical consequence.

\subsection{Modality}

Aristoteles: Syllogisms.  ``things follow from assumptions out of necessity
because these things are so''.  Two side issues: for Aristoteles, the conclusion
must be different from the assumptions, and all assumptions must hold.  Today,
we no longer think that is important.  The main point here: necessity.  We state
two informal definitions of logical consequence:

(M) $\varphi$ is a logical consequence of $\Gamma$ if it isn't possible for all
of $\Gamma$ to be true and $\varphi$ to be false.

(PW) $\varphi$ is a logical consequence of $\Gamma$ if $\varphi$ is true in
every possible world in which all of $\Gamma$ is true.

The difference between (M) and (PW) is that the latter explicitly mentions
possible worlds.  They have in common that the conclusion must follow
necessarily.

Consider the following two examples:  ``Bill is heavier than Joe'' is a
consequence of ``Joe is lighter than Bill''. And ``Caius has an immortal soul''
is a consequence of ``Caius is human''.  Both of them are vaild consequences
according to (M) and (PW), but do we want them to be valid consequences?
\emph{We silently assume indubitable connections between heavier / lighter and
being human / having a soul.  These are outside facts.}  Silent assumptions are
always a bad thing, so let's try to take them into account in the next section.

\subsection{Semantics}

Another notion of logical consequence:

(S) $\varphi$ is a logical consequence of $\Gamma$ if the truth of the members
of $\Gamma$ guarantees the truth of $\varphi$ in virtue of the meanings of the
terms in those sentences.

\emph{There are two differnces between (M) and (S): first, the meaning of terms
is now mentioned explicitly.  In a sense, it becomes part of the assumptions
now.  Second, Shapiro prases it differently. Instead of ``\ldots it is not
possible \ldots'', he now says ``\ldots guarantees \ldots''.  Is that of any
relevance? I don't think so but am not sure.}

(S) is a refinement of (M).  The meaning of the term ``being human'' does not
necessarily mean that humans have immortal souls.  We have therefore broken the
outside link between being human and having soul, so according to (S), the Caius
example is no longer a valid consequence.

\emph{The only thing we did to go from (M) to (S) is to drag silent assumptions
into the spotlight.  How does this break the link between being human and having
soul? If there is no link under (S), then there wasn't any to begin with in
(M).}

\emph{The point seems to be: In (M), arbitrary outside knowledge is allowed. In
(S), only outside knowledge is allowed that is closely tied to the immediate
meaning of the terms.}

\subsection{Form}


\bibliographystyle{plain}
\bibliography{logic}

\end{document}

% vim: textwidth=80
