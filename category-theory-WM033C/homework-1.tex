\documentclass[a4paper]{article}
%\usepackage{mathptmx}
\usepackage{amsmath}
\usepackage{amssymb}
\usepackage{amsthm}
\usepackage[retainorgcmds]{IEEEtrantools}
%\usepackage[square]{natbib}

% needs debian package texlive-math-extra
\usepackage{stmaryrd} % for \llbracket, \rrbracket (scott brackets)

\usepackage{enumerate} % for enumerations with letters (a), (b), etc.

\usepackage{tikz}
\usetikzlibrary{positioning}
\usetikzlibrary{matrix}

\newcommand{\arr}{\rightarrow}
%\newcommand{\todo}[1]{\bigskip \noindent \emph{todo: #1}}
%\newcommand{\todo}[1]{}
%\newcommand{\semantics}[1]{\llbracket #1 \rrbracket}

\newcommand{\type}{\!:\!}
%\newcommand{\qed}[1]{$\square (#1)}

\begin{document}

\title{Category Theory Homework 1}
\author{Markus Klinik}

\section{~}

\begin{enumerate}[(a)]
  \item Let $f \type X \arr Y$ and $g, g' \type Y \arr X$ both be inverses of
  $f$. Want: $g = g'$. Have: $g \circ f = 1_X$ and $f \circ g' = 1_Y$.

  \begin{IEEEeqnarray*}{rCl}
  g & = & g \circ 1_Y \\
    & = & g \circ (f \circ g') \\
    & = & (g \circ f) \circ g' \\
    & = & 1_X \circ g' \\
    & = & g' % '
  \end{IEEEeqnarray*}
  \qed{(a)}


  \item Show that $1_X \type X \arr X$ is an isomorphism. To prove:
  $\exists f \type X \arr X$ s.t. $1_X \circ f = 1_X = f \circ id_X$.  Take $f =
  1_X$. $1_X \circ 1_X = 1_X$.
  \qed{(b)}


  \item Show that if $f \type X \arr Y, g \type Y \arr Z$ are isos, $g \circ f$
  is an iso.

  Have: \begin{itemize}
    \item $f^{-1} \type Y \arr X$ such that $f^{-1} \circ f = 1_X$ and $f \circ
    f^{-1} = 1_Y$.
    \item $g^{-1} \type Z \arr Y$ such that $g^{-1} \circ g = 1_Y$ and $g \circ
    g^{-1} = 1_Z$.
    \end{itemize}

  Want: $h \type Z \arr X$ such that $h \circ (g \circ f) = 1_X$ and $(g \circ
  f) \circ h = 1_Z$. Take $h = (f^{-1} \circ g^{-1})$.

  \begin{IEEEeqnarray*}{rCl}
  h \circ (g \circ f) & = & (f^{-1} \circ g^{-1}) \circ (g \circ f) \\
    & = & f^{-1} \circ (g^{-1} \circ g) \circ f \\
    & = & f^{-1} \circ 1_Y \circ f \\
    & = & f^{-1} \circ f \\
    & = & 1_X
  \end{IEEEeqnarray*}
  \begin{IEEEeqnarray*}{rCl}
  (g \circ f) \circ h & = & (g \circ f) \circ (f^{-1} \circ g^{-1}) \\
    & = & g \circ (f \circ f^{-1}) \circ g^{-1} \\
    & = & g \circ 1_Y \circ g^{-1} \\
    & = & g \circ g^{-1} \\
    & = & 1_Z
  \end{IEEEeqnarray*}
  \qed{(c)}


  \item Aut(X) is the set of isomorphisms $X \arr X$. It has the structure of a
  group, where addition is composition, 0 is $1_X$ and inverses are the inverses
  of isos.
  \qed{(d)}

\end{enumerate}

\section{(Awodey 1.9.3)}

\begin{enumerate}[(a)]

  \item Show that in Set the isomorphisms are exactly the bijective functions.
  To prove:
  \begin{enumerate}[(i)]
    \item $f$ is an isomorphism in Set $\implies f$ is a bijective function.
    \item $f$ is a bijective function $\implies f$ is an isomorphism in Set.
  \end{enumerate}

  Proof of (i):
    Let $f \type X \arr Y$ be an isomorphism. This gives us an $f^{-1}
    \type Y \arr X$ such that $f \circ f^{-1} = 1_Y$ and $f^{-1} \circ f = 1_X$.
    Want: f is bijective i.e.~injective and surjective.

    Proof of injectivity: To prove: $\forall a,b. f(a) = f(b) \implies a = b$.
    Let $a,b \in X$ such that $f(a) = f(b)$. Want: $a = b$. Consider $a,b$ as
    arrows $1 \arr X$.  Then $f(a) = f \circ a$ and $f(b) = f \circ b$. Have: $f
    \circ a = f \circ b$.  But then
    \begin{IEEEeqnarray*}{rCl}
    a & = & 1_X \circ a \\
      & = & (f^{-1} \circ f) \circ a \\
      & = & f^{-1} \circ (f \circ a) \\
      & = & f^{-1} \circ (f \circ b) \\
      & = & (f^{-1} \circ f) \circ b \\
      & = & 1_X \circ b \\
      & = & b
    \end{IEEEeqnarray*} \qed{(injectivity)}

    Proof of surjectivity: To prove: $\forall y \in Y. \exists x \in X. f \circ
    x = y$.  Let $y \in Y$. Want: some $x \in X$ such that $f \circ x = y$.  But
    we know that $y = 1_Y \circ y = (f \circ f^{-1}) \circ y$. Take $x = f^{-1}
    \circ y$.  Such an $x$ exists because $f^{-1}$ is a function. Then we have:
    \begin{IEEEeqnarray*}{rCl}
    f \circ x & = & f \circ (f^{-1} \circ y) \\
      & = & (f \circ f^{-1}) \circ y \\
      & = & 1_Y \circ y \\
      & = & y
    \end{IEEEeqnarray*}
    \qed{(surjectivity)} \\
    \qed{(i)}

  Proof of (ii): Let $f$ be an injective function. Want: $f$ is an isomorphism
  in Set, i.e.~$\exists f^{-1}. f \circ f^{-1} = 1_Y$ and $f^{-1} \circ f =
  1_X$.  Take $f^{-1}(y)$ to be the unique $x \in X$ such that $f(x) = y$. Such
  an x exists and is unique because f is surjective and injective.  Now, let $x
  \in X$. Want: $f^{-1} \circ f = 1_X$. Consider:
  \begin{IEEEeqnarray*}{rCl}
  (f^{-1} \circ f)(x) & = & f^{-1}(f(x)) \\
    & = & \text{the unique $x \in X$ such that }f(x) = f(x) \\
    & = & x
  \end{IEEEeqnarray*}

  Let $y \in Y$. Want: $f \circ f^{-1} = 1_Y$. Consider:
  \begin{IEEEeqnarray*}{rCl}
  (f \circ f^{-1})(y) & = & f(f^{-1}(y)) \\
    & = & f( \text{the unique $x \in X$ such that } f(x) = y ) \\
    & = & y
  \end{IEEEeqnarray*}
  \qed{(ii)}\\
  \qed{2(a)}


  \item To prove: $f \type X \arr Y$ is an iso in \textbf{Mon} iff $f$ is a
  bijective monoid homomorphism.

  Proof of $\Rightarrow$:  Let $f$ be an iso in \textbf{Mon}.  Then $f$ is a
  monoid homomorphism by definition, and it is a bijective function by the same
  argument as in 2(a).  \qed{$\Rightarrow$}

  Proof of $\Leftarrow$: Let $f$ be a bijective monoid homomorphism. Want: $f$
  is an iso in \textbf{Mon}, i.e. $\exists f^{-1}$ such that \begin{enumerate}[(i)]
  \item $f \circ f^{-1} = 1_Y$
  \item $f^{-1} \circ f = 1_X$
  \item $f^{-1}$ is a monoid homomorphism.
  \end{enumerate}

  As $f$ is a bijective function, we can define $f^{-1}$ just as in 2(a):
  $f^{-1}(y) = $ the unique x such that $f(x) = y$.  The proofs of (i) and (ii)
  are then the same as in 2(a).  It remains to show that this $f^{-1}$ is indeed
  a monoid homomorphism.  To prove: it preserves 0 and +.

  Proof of 0: Consider:
  \begin{IEEEeqnarray*}{rCl}
  f^{-1}(0_Y) & = & f^{-1}(f(0_X)) \\
    & = & (f^{-1} \circ f)(0_X) \\
    & = & id_X(0_X) \\
    & = & 0_X
  \end{IEEEeqnarray*}
  \qed({0})

  Proof of +: Let $x,y \in Y$. Consider:
  \begin{IEEEeqnarray*}{rCl}
  f^{-1}(x) + f^{-1}(y) & = & 1_X(f^{-1}(x) + f^{-1}(y)) \\
    & = & (f^{-1} \circ f)(f^{-1}(x) + f^{-1}(y)) \\
    & = & f^{-1}(f(f^{-1}(x) + f^{-1}(y))) \\
    & = & f^{-1}(f(f^{-1}(x)) + f(f^{-1}(y))) \\
    & = & f^{-1}((f \circ f^{-1}(x) + (f \circ f^{-1})(y)) \\
    & = & f^{-1}(1_Y(x) + 1_Y(y)) \\
    & = & f^{-1}(x + y)
  \end{IEEEeqnarray*}
  \qed{(+)}\\
  \qed{$\Leftarrow$}\\
  \qed{2(b)}


  \item To show that in \textbf{Poset} the isomorphisms and bijective
  homomorphisms don't coincide, we give a counterexample of two posets and a
  bijective function between them where one direction is monotone but the other
  isn't.

  Consider the following posets $X = (\{x, y\}, \{x \leq x, y \leq y\})$ and $Y
  = (\{a, b\}, \{a \leq b, a \leq a, b \leq b\})$, and the function $f \type X
  \arr Y$, $x \mapsto a$, $y \mapsto b$.  Now $f$ is trivially a monotone
  function, because the premise of $x \leq y \implies f(x) \leq f(y)$ is never
  fulfilled, except for the diagonal. It is also bijective.  The inverse of $f$
  fails to be monotone, because $a \leq b$ but not $x \leq y$.  \qed{(c)}\\
  \qed{(2)}

\end{enumerate}


\section{ }
\begin{enumerate}[(a)]

  \item To prove: a monoid homomorphism $\mathbb{N} \arr \mathbb{Z}$ is
  determined by what it does with 1. Let $f, g \type \mathbb{N} \arr \mathbb{Z}$
  be monoid homomorphisms such that $f(1) = g(1)$.  To prove: $f = g$.  Want:
  $\forall n \in \mathbb{N}. f(n) = g(n)$.  Let $n \in \mathbb{N}$.  Consider:
  \begin{IEEEeqnarray*}{rCl}
  f(n) & = & f(1 + \ldots + 1) \\
    & = & f(1) + \ldots + f(1) \\
    & = & g(1) + \ldots + g(1) \\
    & = & g(1 + \ldots + 1) \\
    & = & g(n)
  \end{IEEEeqnarray*}
  \qed{(a)}


  \item TODO

\end{enumerate}

\end{document}

% vim: textwidth=80
