\documentclass{beamer}

\beamertemplatenavigationsymbolsempty

\mode<presentation>
{
  \usetheme{default}
}

\usepackage[english]{babel}
\usepackage[latin1]{inputenc}
\usepackage{bussproofs}
\usepackage[retainorgcmds]{IEEEtrantools}

% needs debian package texlive-math-extra
\usepackage{stmaryrd} % for \llbracket, \rrbracket

\usepackage{times}
\usepackage[T1]{fontenc}
% Or whatever. Note that the encoding and the font should match. If T1
% does not look nice, try deleting the line with the fontenc.

\usepackage{tikz}
\usetikzlibrary{positioning}
\usetikzlibrary{calc}
\usetikzlibrary{matrix}

\title
{A Type Checker for SPL, the Simple Programming Language}

\author
{Markus~Klinik}

\institute[Radboud University Nijmegen]
{
  Radboud University Nijmegen
}

\date
{Compiler Construction 2013}


\newcommand{\arr}{\rightarrow}
\newcommand{\Arr}{\Rightarrow}
\newcommand{\semantics}[1]{\llbracket #1 \rrbracket}
\newcommand{\semanticsFd}[1]{\semantics{#1}_{F\delta}}

\begin{document}

\begin{frame}
  \titlepage
\end{frame}

\begin{frame}{Features}

  \begin{itemize}
    \item Type inference
    \item Polymorphic functions
    \item Monomorphic variables
    \item Mutual recursion
    \item Higher-order functions
  \end{itemize}

\end{frame}

\begin{frame}{Meta}

  \begin{itemize}
    \item Implementation language: Haskell
    \item Time to implement: 80 hours
    \item Lines of code: 520
  \end{itemize}

\end{frame}


\begin{frame}[fragile]{Example}

\begin{verbatim}
fun f(x x, y y)
{
  print(x + 1);
  return y;
}
\end{verbatim}

Function $f$ gets type $\forall a . \text{Int} \arr a \arr a$

\end{frame}


\begin{frame}[fragile]{Problem: Assignments}

\onslide<+->

Question: What is the type of $f$?

\begin{verbatim}
x x = [];
Void f(y y) {
  x = y;
}
\end{verbatim}

\onslide<+->

It must \emph{not} be: $\forall a . [a] \arr \text{Void}$, for otherwise
this would be possible:

\begin{verbatim}
Void main() {
  f(1:[]);
  f(True:[]);
}
\end{verbatim}

\end{frame}


\begin{frame}{Solution: Be Less General}

Answer: Adopt the \emph{value restriction}.

\begin{itemize}
  \item Don't generalize the type of variables
  \item Don't generalize type variables in function types that are free
  in the environment
\end{itemize}

See example in a minute.

\end{frame}


\begin{frame}[fragile]{Mutual recursion}

\onslide<+->

Problems with mutually recursive functions:

\begin{itemize}
  \item Functions use identifiers that are defined later
  \item Functions must be unified together
\end{itemize}

\begin{verbatim}
fun f(x x, y y) { return g(y, x); }
fun g(x x, y y) { return f(x, y); }
\end{verbatim}

Should give $\forall ab . a \arr a \arr b$ and not $\forall abc . a
\arr b \arr c$

\onslide<+->

\vspace{1em}

Solution:

\begin{itemize}
  \item Programmer must tell the compiler which functions are mutually
  recursive.
\end{itemize}

\end{frame}


\begin{frame}{Examples}
\end{frame}


\end{document}
