\documentclass[11pt,a4paper]{article}
\usepackage[utf8x]{inputenc}
\usepackage{amsmath}
\usepackage{amssymb}
\usepackage{bussproofs}
\usepackage{parskip}
\begin{document}

\def\arr{\rightarrow}
\def\lam{\lambda}

\section*{Exercise 1}

a. Show that $(B \arr (A \arr B) \arr C) \arr B \arr C$ is a tautology.
\begin{prooftree}
                \AxiomC{$[B \arr (A \arr B) \arr C ^x]$}
                         \AxiomC{$[B^y]$}
                  \RightLabel{\scriptsize $E\arr$}
                  \BinaryInfC{$(A\arr B)\arr C$}
                                           \AxiomC{$[B^y]$}
                                                \RightLabel{\scriptsize $I[z] \arr$}
                                         \UnaryInfC{$A \arr B$}
                               \RightLabel{\scriptsize $E \arr$}
                      \BinaryInfC{$C$}
                          \RightLabel{\scriptsize $I[y] \arr$}
            \UnaryInfC{$B \arr C$}
                  \RightLabel{\scriptsize $I[x] \arr$}
\UnaryInfC{$(B \arr (A \arr B) \arr C) \arr B \arr C$}
\end{prooftree}

b. Give the type derivation in simply typed $\lambda$-calculus corresponding to the proof of 1a.
\begin{prooftree}
\def\fCenter{\ \vdash\ }

\AxiomC{}
\UnaryInfC{$\Gamma \vdash x : B \arr (A \arr B) \arr C$}
\AxiomC{}
\UnaryInfC{$\Gamma \vdash y : B$}
\BinaryInfC{$\Gamma \vdash x y : (A \arr B) \arr C$}

   \AxiomC{}
   \UnaryInfC{$\Gamma, z:A \vdash y : B$}
   \UnaryInfC{$\Gamma \vdash \lam z : A . y : A \arr B$}
\BinaryInf$\Gamma \fCenter x y (\lam z:A . y) : C$
\UnaryInf$x:B \arr (A \arr B) \arr C \fCenter \lam y:B . x y (\lam z : A . y) : B \arr C$
\UnaryInfC{$\vdash \lam x:B \arr (A \arr B) \arr C .
                   \lam y:B .
                     x y (\lam z:A.y) : (B \arr (A \arr B) \arr C) \arr B \arr C
$}
\end{prooftree}
where $\Gamma = x:B \arr (A \arr B) \arr C, y:B$.

\section*{Exercise 2}

a. Show that $(A \arr A \arr B) \arr A \arr B$ is a tautology.
\begin{prooftree}
\AxiomC{$[A \arr A \arr B^x]$}
\AxiomC{$[A^y]$}
\RightLabel{\scriptsize $E\arr$}
\BinaryInfC{$A \arr B$}
\AxiomC{$[A^y]$}
\RightLabel{\scriptsize $E\arr$}
\BinaryInfC{$B$}
\RightLabel{\scriptsize $I[y]\arr$}
\UnaryInfC{$A \arr B$}
\RightLabel{\scriptsize $I[x]\arr$}
\UnaryInfC{$(A \arr A \arr B) \arr A \arr B$}
\end{prooftree}

b. Give the type derivation in simply typed $\lambda$-calculus corresponding to the proof of 2a.
\begin{prooftree}
\AxiomC{}
\UnaryInfC{$\Gamma \vdash x : A \arr A \arr B$}
\AxiomC{}
\UnaryInfC{$\Gamma \vdash y : A$}
\BinaryInfC{$\Gamma \vdash x y : A \arr B$}
\AxiomC{}
\UnaryInfC{$\Gamma \vdash y : A$}
\BinaryInfC{$\Gamma \vdash x y y : B$}
\UnaryInfC{$x:(A \arr A \arr B) \vdash \lam y:A . x y y : A \arr B$}
\UnaryInfC{$\vdash \lam x:(A \arr A \arr B) . \lam y:A . x y y : (A \arr A \arr B) \arr A \arr B$}
\end{prooftree}
where $\Gamma = x:(A \arr A \arr B), y:A$.

\section*{Exercise 3}

a. Show that the formula $((A \arr B \arr A) \arr B) \arr B$ is a tautology of
first-order minimal propositional logic.
\begin{prooftree}
\AxiomC{[$(A \arr B \arr A) \arr B^x]$}

\AxiomC{$[A^y]$}
\RightLabel{\scriptsize $I[z]\arr$}
\UnaryInfC{$B \arr A$}
\RightLabel{\scriptsize $I[y]\arr$}
\UnaryInfC{$A \arr B \arr A$}
\RightLabel{\scriptsize $E\arr$}
\BinaryInfC{$B$}
\RightLabel{\scriptsize $I[x]\arr$}
\UnaryInfC{$((A \arr B \arr A) \arr B) \arr B$}
\end{prooftree}

b. Give the type derivation in simply typed $\lam$-calculus corresponding to the
proof of 3a.
\begin{prooftree}
\AxiomC{}
\UnaryInfC{$\Gamma \vdash x : (A \arr B \arr A) \arr B$}

\AxiomC{}
\UnaryInfC{$\Gamma,y:A,z:B \vdash y : A$}
\UnaryInfC{$\Gamma,y:A \vdash \lam z:B.y : B \arr A$}
\UnaryInfC{$\Gamma \vdash (\lam y:A . \lam z:B . y) : A \arr B \arr A$}
\BinaryInfC{$\Gamma \vdash x (\lam y:A . \lam z:B . y) : B$}
\UnaryInfC{$\vdash \lam x : (A \arr B \arr A) \arr B . x (\lam y:A . \lam z:B . y) : ((A \arr B \arr A) \arr B) \arr B$}
\end{prooftree}
where $\Gamma = x : (A \arr B \arr A) \arr B$.

\section*{Exercise 4}

a. Show that $((A \arr B) \arr C \arr D) \arr C \arr B \arr D$ is a tautology.
\begin{prooftree}

\AxiomC{[$(A \arr B) \arr C \arr D^x]$}

\AxiomC{$[B^z]$}
\RightLabel{\scriptsize $I[w]\arr$}
\UnaryInfC{$A \arr B$}
\RightLabel{\scriptsize $E\arr$}
\BinaryInfC{$C \arr D$}

\AxiomC{$[C^y]$}
\RightLabel{\scriptsize $E\arr$}
\BinaryInfC{$D$}
\RightLabel{\scriptsize $I[z]\arr$}
\UnaryInfC{$B \arr D$}
\RightLabel{\scriptsize $I[y]\arr$}
\UnaryInfC{$C \arr B \arr D$}
\RightLabel{\scriptsize $I[x]\arr$}
\UnaryInfC{$((A \arr B) \arr C \arr D) \arr C \arr B \arr D$}
\end{prooftree}

b. Give the type derivation in simply typed $\lam$-calculus corresponding to the
proof of 4a.
\begin{prooftree}
\def\fCenter{\ \vdash\ }

\AxiomC{}
\UnaryInfC{$\Gamma \vdash x : (A \arr B) \arr C \arr D$}

\AxiomC{}
\UnaryInfC{$\Gamma, w:A \vdash z : B$}
\UnaryInfC{$\Gamma \vdash (\lam w:A.z) : A \arr B$}
\BinaryInfC{$\Gamma \vdash x(\lam w:A.z) : C \arr D$}

\AxiomC{}
\UnaryInfC{$\Gamma \vdash y : C$}

\BinaryInf$\Gamma \fCenter x (\lam w:A.z) y : D$
\UnaryInf$x : (A \arr B) \arr C \arr D, y:C \fCenter \lam z:B . x(\lam w:A.z)y : B \arr D$
\UnaryInf$x : (A \arr B) \arr C \arr D \fCenter \lam y:C . \lam z:B . x(\lam w:A.z)y : C \arr B \arr D$

\UnaryInfC{$\vdash \lam x : (A \arr B) \arr C \arr D . \lam y : C \lam z : B .
  x (\lam w:A.z) y :
  ((A \arr B) \arr C \arr D) \arr C \arr B \arr D$}
\end{prooftree}
where $\Gamma = x : (A \arr B) \arr C \arr D, y:C, z:B$.

\section*{Exercise 5}

a. What is the definition of a detour in a natural deduction proof?

A detour is an arrow introduction followed immediately by an arrow elimination.

b. Give a proof of $A \arr A \arr A$ in first-order minimal propositional logic that
contains a detour.
\begin{prooftree}
\AxiomC{$[A^z]$}
\RightLabel{\scriptsize $I[z]\arr$}
\UnaryInfC{$A \arr A$}

\AxiomC{$[A^x]$}
\RightLabel{\scriptsize $E\arr$}
\BinaryInfC{$A$}
\RightLabel{\scriptsize $I[y]\arr$}
\UnaryInfC{$A \arr A$}

\RightLabel{\scriptsize $I[x]\arr$}
\UnaryInfC{$A \arr A \arr A$}
\end{prooftree}

c. Give the $\lam$-term that corresponds to the proof of 5b.
Which part corresponds to the detour?
Give the normal form of the $\lam$-term.

The $\lam$-term
\begin{equation*}
  \lam x:A . \lam y:A . (\lam z:A.z) x
\end{equation*}
corresponds to the proof of 5b. The part
\begin{equation*}(\lam z:A.z) x\end{equation*} corresponds to the
detour. The normal form of this term is
\begin{equation*}
  \lam x:A . \lam y:A . x
\end{equation*}

\end{document}
