\documentclass{beamer}

% required for QuaternaryInfC, something with dimension registers
\usepackage{etex}

\beamertemplatenavigationsymbolsempty

\mode<presentation>
{
  \usetheme{default}
}

\usepackage[english]{babel}
\usepackage[latin1]{inputenc}

\usepackage{bussproofs}
\def\defaultHypSeparation{\hskip .1in}

% needs debian package texlive-math-extra
\usepackage{stmaryrd} % for \llbracket, \rrbracket

% for IEEEeqnarray, needs debian package texlive-publishers
\usepackage[retainorgcmds]{IEEEtrantools}

\usepackage{times}
\usepackage[T1]{fontenc}
% Or whatever. Note that the encoding and the font should match. If T1
% does not look nice, try deleting the line with the fontenc.

\usepackage{tikz}
\usetikzlibrary{positioning}
\usetikzlibrary{calc}

\title
{Syntax Directed Type Checking for Pure~Type~Systems}

\author
{Markus~Klinik}

\institute[Radboud University Nijmegen] % (optional, but mostly needed)
{
  Radboud University Nijmegen
}

\date
{Type Theory and Proof Assistants 2012}


\newcommand{\arr}{\rightarrow}
\newcommand{\Arr}{\Rightarrow}

% church-style (explicit) abstraction. We adjust the spacing around the colon and dot
\newcommand{\church}[4]{#1 #2\!:\!#3\,.\,#4}

% something is of type something  x : A, we adjust the spacing
\newcommand{\oftype}[2]{#1\!:\!#2}

\begin{document}

\begin{frame}
  \titlepage
\end{frame}

\begin{frame}{Outline}
  \tableofcontents
  % You might wish to add the option [pausesections]
\end{frame}


% Structuring a talk is a difficult task and the following structure
% may not be suitable. Here are some rules that apply for this
% solution:

% - Exactly two or three sections (other than the summary).
% - At *most* three subsections per section.
% - Talk about 30s to 2min per frame. So there should be between about
%   15 and 30 frames, all told.

% - A conference audience is likely to know very little of what you
%   are going to talk about. So *simplify*!
% - In a 20min talk, getting the main ideas across is hard
%   enough. Leave out details, even if it means being less precise than
%   you think necessary.
% - If you omit details that are vital to the proof/implementation,
%   just say so once. Everybody will be happy with that.

\section{What?}

\subsection{What is Type Checking?}


\begin{frame}{What is Type Checking?}

  \begin{itemize}
    \item
      Type Checking: Given $\Gamma$, $M$ and $A$, decide whether the judgement
      $\Gamma \vdash M : A$ is derivable
    \item
      Type Synthesis: Given $\Gamma$ and $M$, compute an $A$ such that $\Gamma
      \vdash M : A$
    \item
      We need a type system
      \begin{itemize}
        \item
          Syntax
        \item
          Typing rules
      \end{itemize}
  \end{itemize}

\end{frame}


\subsection{What Does ``Syntax Directed'' Mean?}


\begin{frame}{What Does ``Syntax Directed'' Mean?}

  \begin{center}
  $M ::= x\ |\ M M\ |\ \lambda x . M$
  \end{center}

  \begin{IEEEeqnarray*}{rCl}
    FV(x)   & = & \{x\} \\
    FV(M N) & = & FV(M) \cup FV(N) \\
    FV(\lambda x . M) & = & FV(M) \textbackslash \{x\}
  \end{IEEEeqnarray*}

\end{frame}

\begin{frame}[fragile]{A Syntax Directed Function}

  \small{\begin{verbatim}
data Term
  = Var Char
  | App Term Term
  | Abs Char Term

freeVars :: Term -> [Char]
freeVars (Var v)   = [v]
freeVars (App l r) = freeVars l `union` freeVars r
freeVars (Abs v t) = filter (/= v) (freeVars t)
  \end{verbatim}}

\end{frame}


\subsection{What are PTSs?}

\begin{frame}{What are PTSs?}

  \begin{itemize}
    \item
      Generalized type systems
    \item
      $\lambda$-zoo $\arr$ $\lambda$-cube $\arr$ PTSs
    \item
      Identical syntax
    \item
      Parametrized typing rules
  \end{itemize}

\end{frame}


\begin{frame}{The Ingredients to Generalized Typing}

  \begin{itemize}
    \item
      Introduce \emph{dependent function types}
    \item
      One syntactic category for terms and types
    \item
      Introduce \emph{sorts}
  \end{itemize}

\end{frame}


\begin{frame}{The $\lambda$-Cube: Ingredients}

  \begin{center}
    $M ::= x \ |\ \star \ |\ \square \ |\ MM \ |\ \church{\lambda}{x}{M}{M} \ |\ \church{\Pi}{x}{M}{M}$
  \end{center}

  \begin{center}
    \begin{prooftree}
     \AxiomC{$M:(\church{\Pi}{x}{\sigma}{\tau(x)})$}
     \AxiomC{$N:\sigma$}
     %\RightLabel{[Unit]}
     \BinaryInfC{$MN:\tau(N)$}
    \end{prooftree}
  \end{center}

\end{frame}

\newcommand{\ruleLambdaCubeAxiom}{
  \begin{prooftree}
     \AxiomC{}
     \LeftLabel{(axiom)}
     \UnaryInfC{$\vdash \star : \square$}
  \end{prooftree}
}

\newcommand{\ruleLambdaCubeStart}{
  \begin{prooftree}
     \AxiomC{$\Gamma \vdash A : s$}
     \AxiomC{$x \notin \Gamma$}
     \LeftLabel{(start)}
     \BinaryInfC{$\Gamma , \oftype{x}{A} \vdash \oftype{x}{A}$}
  \end{prooftree}
}

\newcommand{\ruleLambdaCubeWeakening}{
  \begin{prooftree}
    \AxiomC{$\Gamma \vdash \oftype{M}{B}$}
    \AxiomC{$\Gamma \vdash \oftype{A}{s}$}
    \AxiomC{$x \notin \Gamma$}
    \LeftLabel{(weakening)}
    \TrinaryInfC{$\Gamma , \oftype{x}{A} \vdash \oftype{M}{B}$}
  \end{prooftree}
}

\newcommand{\ruleLambdaCubeApplication}{
  \begin{prooftree}
    \AxiomC{$\Gamma \vdash \oftype{M}{(\church{\Pi}{x}{A}{B})}$}
    \AxiomC{$\Gamma \vdash \oftype{N}{A}$}
    \LeftLabel{(application)}
    \BinaryInfC{$\Gamma \vdash \oftype{MN}{[N/x]B}$}
  \end{prooftree}
}

\newcommand{\ruleLambdaCubeAbstraction}{
  \begin{prooftree}
    \AxiomC{$\Gamma , \oftype{x}{A} \vdash \oftype{M}{B}$}
    \AxiomC{$\Gamma \vdash \oftype{(\church{\Pi}{x}{A}{B})}{s}$}
    \AxiomC{$x \notin \Gamma$}
    \LeftLabel{(abstraction)}
    \TrinaryInfC{$\Gamma \vdash
      \oftype{(\church{\lambda}{x}{A}{M})}
             {(\church{\Pi}{x}{A}{B})}$}
  \end{prooftree}
}

\newcommand{\ruleLambdaCubeConversion}{
  \begin{prooftree}
    \AxiomC{$\Gamma \vdash \oftype{M}{A}$}
    \AxiomC{$A =_{\beta} B$}
    \AxiomC{$\Gamma \vdash \oftype{B}{s}$}
    \LeftLabel{(conversion)}
    \TrinaryInfC{$\Gamma \vdash \oftype{M}{B}$}
  \end{prooftree}
}

\newcommand{\ruleLambdaCubeProduct}{
  \begin{prooftree}
    \AxiomC{$\Gamma \vdash \oftype{A}{s_1}$}
    \AxiomC{$\Gamma , \oftype{x}{A} \vdash \oftype{B}{s_2}$}
    \AxiomC{$x \notin \Gamma$}
    \LeftLabel{(product)}
    \TrinaryInfC{$\Gamma \vdash \oftype{(\church{\Pi}{x}{A}{B})}{s_2}$}
  \end{prooftree}
}

\begin{frame}{$\lambda$-Cube Typing Rules}

  \begin{columns}

    \begin{column}{0.5\textwidth}
      \ruleLambdaCubeAxiom
    \end{column}

    \begin{column}{0.5\textwidth}
      \ruleLambdaCubeStart
    \end{column}

  \end{columns}

  \ruleLambdaCubeWeakening

  \ruleLambdaCubeApplication

  \ruleLambdaCubeAbstraction

  \ruleLambdaCubeProduct

  \ruleLambdaCubeConversion

\end{frame}


\begin{frame}{$\lambda$-Cube Rules: Axiom, Start}

  \ruleLambdaCubeAxiom

  \ruleLambdaCubeStart

\end{frame}


\begin{frame}{$\lambda$-Cube Rules: Weakening}

  \ruleLambdaCubeStart

  \ruleLambdaCubeWeakening

\end{frame}


\begin{frame}{$\lambda$-Cube Rules: Application, Abstraction}

  \ruleLambdaCubeApplication

  \ruleLambdaCubeAbstraction

\end{frame}


\begin{frame}{$\lambda$-Cube Rules: Product}

  \ruleLambdaCubeAbstraction

  \ruleLambdaCubeProduct

\end{frame}


\begin{frame}{$\lambda$-Cube Rules: Conversion}

  \ruleLambdaCubeApplication

  \ruleLambdaCubeConversion

\end{frame}


\begin{frame}{$\lambda$-Cube Product Rule Again}

  \ruleLambdaCubeProduct

  \begin{center}
    \begin{tabular}{|c|c|}
    \hline
    $(\star, \star)$ & terms depending on terms \\
    \hline
    $(\square, \star)$ & terms depending on types \\
    \hline
    $(\star, \square)$ & types depending on terms \\
    \hline
    $(\square, \square)$ & types depending on types \\
    \hline
    \end{tabular}
  \end{center}

\end{frame}

\begin{frame}{$\lambda$-Cube Wrap Up}

  A type system in the $\lambda$-cube consists of:

  \begin{itemize}
    \item
      The set of pseudoterms
    \item
      The set of sorts $\mathcal{S} = \{\star$, $\square$\}
    \item
      A set $\mathcal{R} \subseteq \mathcal{S} \times \mathcal{S}$ of \emph{product rules}
    \item
      The typing rules
  \end{itemize}

\end{frame}

\newcommand{\rulePtsProduct}{
  \begin{prooftree}
    \AxiomC{$\Gamma \vdash \oftype{A}{s_1}$}
    \AxiomC{$\Gamma , \oftype{x}{A} \vdash \oftype{B}{s_2}$}
    \AxiomC{$\mathcal{R}(s_1, s_2, s_3)$}
    \AxiomC{$x \notin \Gamma$}
    \LeftLabel{(product)}
    \QuaternaryInfC{$\Gamma \vdash \oftype{(\church{\Pi}{x}{A}{B})}{s_3}$}
  \end{prooftree}
}

\begin{frame}{And Now For Something Slightly Different}

  A pure type system consists of:

  \begin{itemize}
    \item
      The set of pseudoterms
    \item
      An arbitrary set $\mathcal{S}$ of sorts
    \item
      A set $\mathcal{A} \subseteq \mathcal{S} \times \mathcal{S}$ of axioms
    \item
      A set $\mathcal{R} \subseteq \mathcal{S} \times \mathcal{S} \times
      \mathcal{S}$ of product rules
    \item
      The typing rules
  \end{itemize}

  \begin{prooftree}
    \AxiomC{$\mathcal{A}(s_1, s_2)$}
    \LeftLabel{(axiom)}
    \UnaryInfC{$\oftype{s_1}{s_2}$}
  \end{prooftree}

  \rulePtsProduct

\end{frame}


\section{Syntax Directed Type Checking for PTSs}

\subsection{The Plan}

\begin{frame}[fragile]{The Plan}

  \rulePtsProduct

  \small{\begin{verbatim}
data Term
  = Prod Char Term Term
  ...

typeSynthesis :: Env -> Term -> Maybe Term
typeSynthesis gamma (Prod x A B) =
  let s1 = typeSynthesis gamma A
      s2 = typeSynthesis (gamma ++ [(x,A)]) B
  in
    if (rel s1 s2 s3) then s3 else Nothing
  \end{verbatim}}


\end{frame}


\subsection{The Problem}

\begin{frame}[fragile]{The Problem}

  \ruleLambdaCubeConversion

  \smallskip

  \small{\begin{verbatim}
...
typeSynthesis gamma M            = ...
typeSynthesis gamma (Prod x A B) = ...
...
  \end{verbatim}}

\end{frame}


\begin{frame}{The Concern}

  \rulePtsProduct

  \begin{itemize}
    \item
      Derivations are trees
    \item
      Context correctness is checked on every branch
  \end{itemize}

\end{frame}


\subsection{The Solution}

\begin{frame}{The Solution}

  \begin{itemize}
    \item
      Restrict ourselves to certain PTSs
      \begin{itemize}
        \item Functional: $\mathcal{A}$ and $\mathcal{R}$ are right-unique
        \item Strongly normalizing
      \end{itemize}
    \item
      Keep track of context correctness instead of repeatedly checking it
    \item
      Absorb conversion rule into other rules where needed
  \end{itemize}

\end{frame}

\newcommand{\vcdash}{\vdash_{vc}}
\newcommand{\vtdash}{\vdash_{vt}}

\begin{frame}{Maintaining Context Validity}

  \vspace{-0.5em}

  \begin{columns}

    \begin{column}{0.4\textwidth}
      \begin{prooftree}
         \AxiomC{}
         \LeftLabel{(valid-nil)}
         \UnaryInfC{$\vcdash \emptyset$}
      \end{prooftree}
    \end{column}

    \begin{column}{0.6\textwidth}
      \begin{prooftree}
         \AxiomC{$\textcolor{red}{\Gamma \vtdash \oftype{A}{s}}$}
         \AxiomC{$x \notin \Gamma$}
         \LeftLabel{(valid-cons)}
         \BinaryInfC{$\Gamma \vcdash \oftype{x}{A}$}
      \end{prooftree}
    \end{column}

  \end{columns}

  \begin{columns}

    \begin{column}{0.5\textwidth}
      \begin{prooftree}
         \AxiomC{$\textcolor{red}{\vcdash \Gamma}$}
         \AxiomC{$\mathcal{A}(s_1, s_2)$}
         \LeftLabel{(axiom)}
         \BinaryInfC{$\Gamma \vtdash \oftype{s_1}{s_2}$}
      \end{prooftree}
    \end{column}

    \begin{column}{0.5\textwidth}
      \begin{prooftree}
         \AxiomC{$\textcolor{red}{\vcdash \Gamma}$}
         \AxiomC{$\oftype{x}{A} \in \Gamma$}
         \LeftLabel{(var)}
         \BinaryInfC{$\Gamma \vtdash \oftype{x}{A}$}
      \end{prooftree}
    \end{column}

  \end{columns}

  \begin{prooftree}
    \AxiomC{$\Gamma \vtdash \oftype{A}{s_1}$}
    \AxiomC{$\Gamma , \oftype{x}{A} \vtdash \oftype{B}{s_2}$}
    \AxiomC{$\mathcal{R}(s_1, s_2, s_3)$}
    \AxiomC{$x \notin \Gamma$}
    \LeftLabel{(product)}
    \QuaternaryInfC{$\Gamma \vtdash \oftype{(\church{\Pi}{x}{A}{B})}{s_3}$}
  \end{prooftree}

  \begin{prooftree}
    \AxiomC{$\Gamma , \oftype{x}{A} \vtdash \oftype{M}{B}$}
    \AxiomC{$\Gamma \vtdash \oftype{(\church{\Pi}{x}{A}{B})}{s}$}
    \AxiomC{$x \notin \Gamma$}
    \LeftLabel{(abstraction)}
    \TrinaryInfC{$\Gamma \vtdash
      \oftype{(\church{\lambda}{x}{A}{M})}
             {(\church{\Pi}{x}{A}{B})}$}
  \end{prooftree}

  \begin{prooftree}
    \AxiomC{$\Gamma \vtdash \oftype{M}{(\church{\Pi}{x}{A}{B})}$}
    \AxiomC{$\Gamma \vtdash \oftype{N}{A}$}
    \LeftLabel{(application)}
    \BinaryInfC{$\Gamma \vtdash \oftype{MN}{[N/x]B}$}
  \end{prooftree}

  \begin{prooftree}
    \AxiomC{$\Gamma \vtdash \oftype{M}{A}$}
    \AxiomC{$A =_{\beta} B$}
    \AxiomC{$\Gamma \vtdash \oftype{B}{s}$}
    \LeftLabel{(conversion)}
    \TrinaryInfC{$\Gamma \vtdash \oftype{M}{B}$}
  \end{prooftree}

\end{frame}


\begin{frame}{The Conversion Rule}

  \begin{itemize}
    \item
      Foo
    \item
      Bar
    \item
      Baz
  \end{itemize}

\end{frame}


\section*{Summary}

\begin{frame}{Summary}

  \begin{itemize}
  \item
    Conclusion A
  \item
    Conclusion B
  \item
    Conclusion C
  \end{itemize}

\end{frame}


\end{document}


